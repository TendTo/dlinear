\chapter{Software architecture}

\section{Overview}

The software is written in C++ and is formed by two main components: the \textit{dlinear library} and the \textit{pydlinear Python bindings}.
The former is the result of the refactoring of the \dlinearfour project.
The process involved a deep study of the codebase, which led to a delicate but rewarding migration to the latest C++17 standard, as well as the update of most external libraries to their latest version.
All the leftovers from the \dreal project have been removed during the refactoring. 
At the same time, some design patterns, such as Singleton and Visitor, have been introduced to improve the code quality and future extensibility.

\pydlinear is a new addition inspired by the \dreal project to provide a more user-friendly interface to the library and allow faster prototyping in a Python environment.
Furthermore, a customized \textit{benchmarking suite} has been introduced to the project, and the \textit{test suite} has been improved with additional coverage.
Their combination allows us to verify the correctness and measure the application's performance.

The build system has also received a lot of attention.
Instead of relying on a manual installation of all the direct and transitive dependencies, the process is now integrated into \bazel.

See \autoref{tab:changes} for a summary of the changes introduced.

\begin{table}[h]
        \hskip-1.5cm
        \begin{tabular}{l|l}
                \textbf{\dlinearfour}                        & \textbf{\dlinear}                                   \\
                \hline
                \hline
                C++14 standard                               & C++17 standard ([[nodiscard]], std::optional)       \\
                \hline
                \dreal leftovers                             & Removed unused code,                                \\
                                                             & changed namespace to \dlinear                       \\
                \hline
                Manual dependencies building                 & \bazel handles all external dependencies            \\
                \hline
                Old dependency version                       & Updated dependencies to the latest version          \\
                \hline
                Old ezoptionparser                           & Modern argparse                                     \\
                command-line parser library                  & command-line parser library                         \\
                \hline
                Multiple independent classes                 & FormulaVisitor superclass with virtual methods      \\
                to perform formula conversions               &                                                     \\
                \hline
                Multiple functions to initialize             & Singleton instance in charge or the                 \\
                and deinitialize the global state            & global state of the program. Ensures                \\
                of the program with no safety check          & the validity of the initialization/deinitialization \\
                \hline
                The parser is also responsible for handling  & Introduced an abstraction layer through a Solver   \\
                the \gls{smt} solver and outputs the results & class. Easy to extend with multiple parsers         \\
                \hline
                                                             & MPS file format support                             \\
                \hline
                                                             & \pydlinear Python bindings                          \\
                \hline
                                                             & Benchmarking suite with CSV output                  \\
                \hline
        \end{tabular}
        \caption{Summary of the changes made to the original \dlinearfour project}\label{tab:changes}
\end{table}

\subsection*{Filesystem tree}

The project is organized in a structure (\autoref{fig:filesystem}) that is common to find in \texttt{bazel} projects.

The root directory contains the \texttt{WORKSPACE} file, marking the workspace's root.
The \dlinear folder contains the C++ source code of the library.
It is further divided into subfolders, each containing the source code of a specific and primarily isolated component.
The same structure is mimicked in the \texttt{test} directory, where all the unit tests are placed.
The \pydlinear folder contains the configurations needed to create the Python bindings and some tests to ensure their correctness. In contrast, the \texttt{benchmark} folder contains the benchmarking suite used to measure the solver's performance.

Following \texttt{bazel} conventions, the \texttt{third\_party} folder contains the external dependencies.
All the build tools and Starlark rules/macros are stored in the \texttt{tools} package.


\begin{figure}[ht]
        \begin{adjustbox}{width=0.5\textwidth,center}
                \centering
                \begin{forest}
                        pic dir tree,
                        where level=0{}{% folder icons by default; override using file for file icons
                                        directory,
                                },
                        [dlinear
                                                [benchmark \qquad\qquad Benchmarking utilities
                                                ]
                                                [dlinear \qquad\qquad\ \ Sources
                                                                [api
                                                                ]
                                                                [libs
                                                                ]
                                                                [mps
                                                                ]
                                                                [smt2
                                                                ]
                                                                [solver
                                                                ]
                                                                [symbolic
                                                                ]
                                                                [util
                                                                ]
                                                                [main.cpp, file
                                                                ]
                                                ]
                                                [pydlinear \qquad\qquad Python bindings
                                                                [test
                                                                ]
                                                                [pydlinear.cpp, file
                                                                ]
                                                ]
                                                [test \quad\qquad\qquad\qquad Unit tests
                                                ]
                                                [script \qquad\qquad\qquad Utility scripts
                                                ]
                                                [third\_party \qquad\quad\  External dependencies
                                                                [com_github_robotlocomotion_drake
                                                                ]
                                                ]
                                                [tools \qquad\qquad\qquad\  Build tools and Starlark macros
                                                ]
                                                [WORKSPACE.bazel, file
                                                ]
                                ]
                \end{forest}
        \end{adjustbox}
        \caption{Filesystem tree of the project}\label{fig:filesystem}
\end{figure}

\section{Design patterns}
\label{sec:patterns}

When a software project inevitably grows in size and complexity, it becomes increasingly difficult to maintain and extend.
A common approach to combat this trend is following some design patterns, often described as solutions to common problems in software engineering.

As a complex project, \dreal uses various design patterns, like the \textit{pimpl idiom} or the \textit{Composite}, which have been ported to \dlinearfour and \dlinear.
In the refactoring of the project, some patterns have then been either improved (\textit{Visitor}) or introduced (\textit{Singleton}).

\subsection*{Pimpl idiom}

The \textit{pimpl idiom} \cite{man:pimpl} is a technique used to hide the implementation details of a class from the user.
The acronym stands for \textit{Pointer to IMPLementation}, and it is a common practice in C++.
The result is a reduction in compilation time upon changes in the private implementation and avoiding exposing the internal details of a class to the Client (\autoref{dg:pattern-pimpl}).
This pattern can be seen as a specialization of the \textit{Bridge pattern} \cite{book:gof}: the actual implementation of the class can be changed freely, decoupled from the abstraction.
The main difference is that there is usually only one concrete abstraction in the \textit{pimpl idiom}.

\plantuml{diagrams/pattern/pimpl}{Generalized UML class diagram of the Pimpl idiom}{pattern-pimpl}

In \dlinear, the implementation of the solver changes drastically between \texttt{soplex} and \texttt{qsoptex}.
While the interface remains the same, the underlying implementation is chosen based on the configuration at runtime.
Therefore, the \texttt{Context} class, instantiated when the satisfiability problem needs to be verified, uses the \textit{pimpl idiom} (\autoref{dg:simplified-pimpl}).
To achieve this result, the \texttt{Context} header defines a forward declaration of the \texttt{ContextImpl} class and adds it as a private member of \texttt{Context}.
The \texttt{ContextImpl} class is then defined in its file, not exposed in the public header.
Other classes can still extend from \texttt{ContextImpl}, allowing the implementation of the solver to be changed without affecting the rest of the codebase.

\plantuml{diagrams/simplified/pimpl}{Simplified UML class diagram of the implementation of the Pimpl idiom as it is used within \dlinear}{simplified-pimpl}

\subsection*{Composite and Visitor patterns}

The Composite pattern allows to treat both objects and their compositions uniformly \cite{book:gof}.
It achieves this by defining a common interface that both Leaves and Composites implement (\autoref{dg:pattern-composite}).
The approach dramatically simplifies the interaction with complex data structures, as the difference between a single object and a collection of objects is hidden.

\plantuml[0.6]{diagrams/pattern/composite}{Generalized UML class diagram of the Composite pattern}{pattern-composite}

The intent of the Visitor pattern is to represent an operation to be performed on the elements of an object structure.
The Visitor lets you define a new operation without changing the classes of the elements on which it operates \cite{book:gof}.
A typical application is traversing an object structure and applying an operation to each element based on its type (\autoref{dg:pattern-visitor}).
The ObjectStructure can be a simple collection or a Composite object.

\plantuml{diagrams/pattern/visitor}{Generalized UML class diagram of the Visitor pattern}{pattern-visitor}

The latter is the case of \dlinear, where the \texttt{Formula} class is a Composite object, and the \texttt{FormulaVisitor} is used to traverse the structure and perform operations on each element (\autoref{dg:simplified-visitor}).
By using the pattern, the Client is exposed to a simplified interface, where the only method to call over a \texttt{FormulaVisitor} object is \texttt{Convert} with the input \texttt{Formula}.
The conversion happens internally, with a sequence of recursive calls to the specialized \texttt{Visit} methods.
Using these classes, it is possible to convert an input formula in \gls{nnf}, \gls{cnf} or in a boolean formula, where each linear constraint is substituted with a boolean variable.

\plantuml{diagrams/simplified/visitor}{Simplified UML class diagram of the Visitor pattern as it is used within \dlinear}{simplified-visitor}

\subsection*{Singleton}

The Singleton pattern is used when there is the need to ensure that a class only has one instance and to provide a global point of access to it \cite{book:gof}.
It is one of the most straightforward design patterns and only requires a class with a private constructor, a private static reference to a Singleton object and a public static method to retrieve the instance \autoref{dg:pattern-singleton}.

\plantuml[0.3]{diagrams/pattern/singleton}{Generalized UML class diagram of the Singleton pattern}{pattern-singleton}

This pattern has seen a fair share of criticisms over the years, mainly because it is often used as a global variable in disguise.
However, it still represents a valid solution to make the code cleaner when such a situation is unavoidable.
This is the case in the initialization of the lower and upper bounds infinity constants in \dlinear.
They ensure that at no time during execution an intermediate result can be greater (or smaller) than these global constants.
What complicates matters is the fact that the \texttt{gmp} library allows for the definition of a set of custom functions to be used when performing dynamic memory allocation, reallocation and deallocation with the \texttt{mp\_set\_memory\_functions} function.
It is a feature that \texttt{QSopt\_ex} uses internally.
Whenever the function is called, all the previously created \texttt{gmp} objects are invalidated.
Hence, using the Singleton pattern allows for centralized control over the state of the \texttt{gmp} library, ensuring that the initialization happens only once and in the correct order \autoref{dg:dlinear-singleton}.
Since memory management must be handled manually, the \texttt{Infinity} class also includes a method to free the currently allocated resources, preventing memory leaks.

\plantuml[0.45]{diagrams/dlinear/singleton}{UML class diagram of the Singleton pattern as it is used within \dlinear}{dlinear-singleton}

\section{Parsers}

The first step in the verification process is always parsing the input to extract the relevant information and build the internal representation the solvers will use.
The parser becomes an essential component of the software: it determines which formats are supported and can slow down the application if not implemented correctly.

\dlinearfour supported only the \gls{smt} format, while \dlinear adds support for the \gls{mps} format.
The original implementation was problematic to extend: the \texttt{Driver} class was built around the \texttt{Smt2Parser} class, and the two were tightly coupled.
Furthermore, it was also responsible for interacting with the \gls{sat} and \gls{smt} solvers, making it a monolithic class and diminishing its cohesion.

The solution was to add an abstraction layer between the \texttt{Driver} and the client, introducing the \texttt{Solver} class.
Its job is determining which parser to use based on the input file's extension.
After completing this first step, the \texttt{Context} containing all the relevant information gathered from the input is produced and passed to the solvers.
Finally, the output is returned to the Client.

\subsection*{Smt2 parser}

The original input files \dlinear accepts are in the format standardized by SMT-LIB \cite{docs:smtlib}.
The parser itself is generated by \texttt{bison} and \texttt{flex} and can be divided into two main parts: the lexer and the parser.
The lexer is responsible for tokenizing the input file by matching the input stream with the regular expressions defined in the \texttt{scanner.ll} file (\autoref{code:example.yy}).
The result is then fed to the parser, which uses the grammar defined in the \texttt{parser.yy} file to build an abstract syntax tree (\autoref{code:example.yy}).

Additionally, a \texttt{Driver} class coordinates the two components and provides an interface to the rest of the application, allowing it to start the parsing process and retrieve the result.
Each rule in the grammar is associated with an action, usually a \texttt{Driver}'s method call.

\lstinputlisting[language=flex,frame=single,showstringspaces=false,caption={Simplified example of a rule tokenizing the smt2 input file to reconnize the "check-sat" directive},captionpos=b,label={code:example.ll}]{code/example.ll}

\lstinputlisting[language=yacc,frame=single,showstringspaces=false,caption={Simplified example of a rule parsing the check-sat token and calling the \texttt{Driver} accordingly},captionpos=b,label={code:example.yy}]{code/example.yy}

\subsection*{MPS parser}

One of the most promising fields where \dlinear can be applied is \textit{linear programming}.
The goal is to verify whether a set of linear constraints is satisfiable, not necessarily finding the optimal solution.
\gls{mps} is a common format to store \gls{lp} problems \cite{man:mps}.
It is worth noting that the \gls{mps} standard is often extended by adding custom sections or the support for integer variables.
The syntax is column-oriented, each column represents a variable and each row a constraint.
The file is divided into sections, each starting with a known identifier.
For example, the following \gls{lp} problem is represented in \gls{mps} format as in \autoref{code:example.mps}.

\begin{equation*}
        \begin{aligned}
                 & \text{minimize}   & - x_1 - 2 x_2       \\
                 & \text{subject to} & - x_1 + x_2 \leq 20 \\
                 &                   & x_1 - 3 x_2 \leq 30 \\
                 &                   & x_1 \leq 40         \\
                 &                   & x_1, x_2 \geq 0
        \end{aligned}
\end{equation*}

\lstinputlisting[language=mps,frame=single,showstringspaces=false,caption={Example of an MPS file \cite{man:mps}},captionpos=b,label={code:example.mps}]{code/example.mps}

The parser is generated using \texttt{bison} and \texttt{flex} and it has inherited its structure from the \texttt{smt2} parser.
The grammar is pretty simple: each section can have an arbitrary number of rows (even no row at all), each with a certain number of columns.
It is sufficient to determine the section the parser is in to know which function of the \texttt{Driver} to call.

The information needed to build the constraints is divided into many different sections, so the whole file must be processed before the \texttt{Context} can be created.
Hence, it is not possible to add the constraints incrementally.
Instead, the \texttt{Driver} stores all the building blocks in some temporary maps, which will be used at the end to finalize the \texttt{Context}.

\section{Solvers}

The two solvers supported by \dlinear are \texttt{soplex} and \texttt{qsoptex}.
Both represent the heart of the software.
They are responsible for solving the \gls{lp} part of the problem while being directed by the \texttt{picosat} \gls{sat} solver.

The API presented to the Client hides all the implementation behind the \texttt{Solver} class.
The configuration will determine which solver to use and with what parameters.
While the input file is being parsed, each variable encountered is stored, to be added to the \texttt{Context} object later.
The same is true for all the assertions regarding the problem's constraints, representing the theory's atoms.
Once the parsing has been completed successfully, everything comes together: the \gls{sat} solver evaluates a set of atoms that needs to be satisfied, which are then used to build the \gls{lp} problem to be passed to either \texttt{soplex} or \texttt{qsoptex}.
If the \gls{sat} solver returns \texttt{unsat}, the program halts immediately, for the problem is trivially unsatisfiable.
Otherwise, the output of the \gls{lp} solver will determine the final result.
If the \gls{lp} problem is solved, then \dlinear will return \texttt{sat} and a model.
If the input is found to be infeasible, an explanation containing the unsatisfiable core will be returned.
By negating and integrating it in the input of the \gls{sat} solver, another iteration will be performed, and the process will repeat (\autoref{dg:qsoptex}).

\plantuml{diagrams/sequence/qsoptex}{Simplified sequence diagram of the interactions between the \gls{sat} and theory solvers}{qsoptex}

\clearpage

\section{Python bindings}

As a programming language, Python is widely used in the scientific community.
While its abstractions make it very easy to read and write, this comes at the expense of performance, which is usually abysmal compared to other languages, such as C++.
There are several Python implementations.
The most popular one is \texttt{CPython}, which is also the one this project targets, but some alternatives include \texttt{Jython}, \texttt{IronPython} and \texttt{PyPy} \cite{man:python-implementations}.

The \texttt{CPython} implementation also allows for extensions.
The Python API defines a set of functions, macros and variables that provide access to the aspects of the Python run-time system.
To access them, it is sufficient to include the header file \texttt{Python.h}.
In other words, Python bindings (or extensions) are a way to write C++ code that integrates seamlessly and can called from a Python program.
Instead of using the standard Python API, though, the much more convenient \texttt{pybind11} library is used \cite{man:pybind11}.
The main appeal of \texttt{pybind11} is that it exposes C++ classes and functions to Python with a very minimalistic syntax.
It also provides a rich built-in type conversion system by inferring type information using compile-time introspection.
The user can extend it even further to support custom types.

In the context of \dlinear, the main functionalities of the solver, as well as some of the most valuable data manipulation classes, have been exposed in a Python package called, unsurprisingly, \pydlinear.
The bindings are defined in the \texttt{pydlinear.cpp} file (\autoref{code:example_python_bindings}).
The \texttt{setup.py}, with the configuration options provided by the \texttt{pyproject.toml} file, defines the steps needed for building the package to make it available to the Python interpreter.
While the ideal scenario would be to build the \dlinear library on the target machine, it is possible to include a pre-build version called a \textit{wheel}.
This specialized archive contains all the shared libraries that make up the package and, provided that the architecture of both the distributor's machine and the user's machine is the same, it can be used without any need for compilation by the final user.

\lstinputlisting[language=c++,frame=single,showstringspaces=false,caption={Simplified example of \pydlinear's bindings},captionpos=b,label={code:example_python_bindings}]{code/example\_python\_bindings.cpp}

The package has been uploaded to the \texttt{PyPI} repository, making it available to anyone with a \texttt{Linux x86\_64} machine and \texttt{Python 3.10}.
It can be installed with the command

\begin{lstlisting}[language=bash,frame=single,showstringspaces=false]
$ pip install -i https://test.pypi.org/simple/ pydlinear
\end{lstlisting}
Once installed, the solver's APIs are accessible by importing the \pydlinear package in the script (\autoref{code:pydlinear.py}). \\ \\ \\ \\

\lstinputlisting[language=python,frame=single,showstringspaces=false,caption={Minimal Python program using \pydlinear},captionpos=b,label={code:pydlinear.py}]{code/pydlinear.py}

\section{Targets}

\bazel uses the concept of \textit{targets} to define the building blocks of a workspace.
A target can be a library, an executable, a test, a script or a data file.
What follows is a synopsis of the most essential targets used in \dlinear and their purpose.

\begin{itemize}
        \item \textbf{//dlinear}: supports both \textit{build} and \textit{run} actions.
              It is the project's main target, used by the \texttt{benchmark} and \texttt{test} targets.
              It produces the fully static binary \texttt{dlinea5}, located in the \texttt{bazel-bin/dlinear} folder.
        \item \textbf{//benchmark}: supports both \textit{build} and \textit{run} actions.
              It creates a binary that wraps the \texttt{dlinear} and measures the time it takes to run the program on a set of problems with a given solver and precision.
              Both parameters can be configured using a \texttt{benchmark.conf} file.
              By default, it checks all the files in the \textit{benchmark/smt2} folder.
        \item \textbf{//test/...}: supports the \textit{test} action.
              It runs all the tests in the \dlinear.
              The logs will be saved in the \texttt{bazel-testlogs} folder.
        \item \textbf{//pydlinear}: supports the \textit{build} action.
              It creates the \texttt{pydlinear} extension.
              It can be a dependency within \bazel for other Python targets.
\end{itemize}
