\chapter{Software architecture}

\section{Overview}

The software is written in \texttt{c++}, and it is formed by two main components: the \textit{dlinear library} and the \textit{python bindings}.
The project includes a \textit{benchmarking suite} and a \textit{test suite}, which are used to verify the correctness and measure the application's performance.
The \texttt{python} bindings provide a more user-friendly interface to the library and allow faster prototyping in a \texttt{python} environment.

\subsection*{Filesystem tree}

The project is organized in a structure (\autoref{fig:filesystem}) that is common to find in \texttt{bazel} projects.

The root directory contains the \texttt{WORKSPACE} file, marking the workspace's root.
The \dlinear folder contains the \texttt{c++} source code of the library.
It is further divided into subfolders, each containing the source code of a specific and primarily isolated component.
The same structure is mimicked in the \texttt{test} directory, where all the unit tests are placed.
The \pydlinear folder contains the configurations needed to create the \texttt{python} bindings and some tests to ensure their correctness. In contrast, the \texttt{benchmark} folder contains the benchmarking suite used to measure the solver's performance.

Following \texttt{bazel} conventions, the \texttt{third\_party} folder contains the external dependencies.
All the build tools and Starlark rules/macros are stored in the \texttt{tools} package.


\begin{figure}[ht]
        \begin{adjustbox}{width=0.5\textwidth,center}
                \centering
                \begin{forest}
                        pic dir tree,
                        where level=0{}{% folder icons by default; override using file for file icons
                                        directory,
                                },
                        [dlinear5
                                                [benchmark \qquad\qquad Benchmarking utilities
                                                ]
                                                [dlinear \qquad\qquad\ \ Sources
                                                                [api
                                                                ]
                                                                [libs
                                                                ]
                                                                [smt2
                                                                ]
                                                                [solver
                                                                ]
                                                                [symbolic
                                                                ]
                                                                [util
                                                                ]
                                                                [main.cpp, file
                                                                ]
                                                ]
                                                [pydlinear \qquad\qquad Python bindings
                                                                [test
                                                                ]
                                                                [pydlinear.cpp, file
                                                                ]
                                                ]
                                                [test \quad\qquad\qquad\qquad Unit tests
                                                ]
                                                [script \qquad\qquad\qquad Utility scripts
                                                ]
                                                [third\_party \qquad\quad\  External dependencies
                                                                [com_github_robotlocomotion_drake
                                                                ]
                                                ]
                                                [tools \qquad\qquad\qquad\  Build tools and Starlark macros
                                                ]
                                ]
                \end{forest}
        \end{adjustbox}
        \caption{Filesystem tree of the project}\label{fig:filesystem}
\end{figure}

\section{Design patterns}
\label{sec:patterns}

\subsection*{Pimpl idiom}

The \textit{pimpl idiom} \cite{man:pimpl} is a technique used to hide the implementation details of a class from the user.
The acronym stands for \textit{Pointer to IMPLementation}, and it is a common practice in \texttt{c++}.
The result is a reduction in compilation time upon changes in the private implementation and avoiding exposing the internal details of a class to the client (\autoref{dg:pattern-pimpl}).
This pattern can be seen as a specialization of the \textit{bridge pattern} \cite{book:gof}: the actual implementation of the class can be changed freely, decoupled from the abstraction.
The main difference is that there is usually only one concrete abstraction in the \textit{pimpl idiom}.

\plantuml{diagrams/pattern/pimpl}{Generalized UML diagram of the Pimpl idiom}{pattern-pimpl}

In \dlinear, the implementation of the solver changes drastically between \texttt{soplex} and \texttt{qsoptex}.
While the interface remains the same, the underlying implementation is chosen based on the configuration at runtime.
Therefore, the \texttt{Context} class, instantiated when the satisfiability problem needs to be verified, uses the \textit{pimpl idiom} (\autoref{dg:simplified-pimpl}).
To achieve this result, the \texttt{Context} header defines a forward declaration of the \texttt{ContextImpl} class and adds it as a private member of \texttt{Context}.
The \texttt{ContextImpl} class is then defined in its file, not exposed in the public header.
Other classes can still extend from \texttt{ContextImpl}, allowing the implementation of the solver to be changed without affecting the rest of the codebase.

\plantuml{diagrams/simplified/pimpl}{Simplified UML diagram of the implementation of the Pimpl idiom as it is used within \dlinear}{simplified-pimpl}

\subsection*{Composite and Visitor patterns}

The Composite pattern allows to treat both objects and their compositions uniformly \cite{book:gof}.
It achieves this by defining a common interface that both Leaves and Composites implement (\autoref{dg:pattern-composite}).
The approach dramatically simplifies the interaction with complex data structures, as the difference between a single object and a collection of objects is hidden.

\plantuml{diagrams/pattern/composite}{Generalized UML diagram of the Composite pattern}{pattern-composite}

The intent of the Visitor pattern is to represent an operation to be performed on the elements of an object structure.
The Visitor lets you define a new operation without changing the classes of the elements on which it operates \cite{book:gof}.
A typical application is traversing an object structure and applying an operation to each element based on its type (\autoref{dg:pattern-visitor}).
The ObjectStructure can be a simple collection or a Composite object.

\plantuml{diagrams/pattern/visitor}{Generalized UML diagram of the Visitor pattern}{pattern-visitor}

The latter is the case of \dlinear, where the \texttt{Formula} class is a Composite object, and the \texttt{FormulaVisitor} is used to traverse the structure and perform operations on each element (\autoref{dg:simplified-visitor}).
Through the Visitors, the input formula is converted in \gls{nnf}, \gls{cnf} or in a boolean formula, where each linear constraint is substituted with a boolean variable.

\plantuml{diagrams/simplified/visitor}{Simplified UML diagram of the Visitor pattern as it is used within \dlinear}{simplified-visitor}

\section{Smt2 parser}

The input files \dlinear accepts are in the format standardized by SMT-LIB \cite{docs:smtlib}.
The parser itself is generated by \texttt{bison} and \texttt{flex} and can be divided into two main parts: the lexer and the parser.
The lexer is responsible for tokenizing the input file by matching the input stream with the regular expressions defined in the \texttt{scanner.ll} file (\autoref{code:example.yy}).
The result is then fed to the parser, which uses the grammar defined in the \texttt{parser.yy} file to build an abstract syntax tree (\autoref{code:example.yy}).

Additionally, a \texttt{Driver} class coordinates the two components and provides an interface to the rest of the application, allowing it to start the parsing process and retrieve the result.
Each rule in the grammar is associated with an action, usually a \texttt{Driver}'s method call.

\lstinputlisting[language=flex,frame=single,showstringspaces=false,caption={Simplified example of a rule tokenizing the smt2 input file to reconnize the "check-sat" directive},captionpos=b,label={code:example.ll}]{code/example.ll}

\lstinputlisting[language=yacc,frame=single,showstringspaces=false,caption={Simplified example of a rule parsing the check-sat token and calling the \texttt{Driver} accordingly},captionpos=b,label={code:example.yy}]{code/example.yy}

\section{Solvers}

The two solvers supported by \dlinear are \texttt{soplex} and \texttt{qsoptex}.
Both represent the heart of the software.
They are responsible for solving the \gls{lp} part of the problem while being directed by the \texttt{picosat} \gls{sat} solver.

While the \texttt{smt2} file is being parsed, each variable encountered is stored and added to the \texttt{Context} object.
The same is true for all the assertions regarding the problem's constraints, which represent the theory's atoms.
When the \texttt{check-sat} directive is reached, everything comes together: the \gls{sat} solver evaluates a set of atoms that needs to be satisfied, which are then used to build the \gls{lp} problem to be passed to either \texttt{soplex} or \texttt{qsoptex} (\autoref{dg:qsoptex}).

\plantuml{diagrams/sequence/qsoptex}{Simplified sequence diagram of the interactions between the \gls{sat} and theory solvers}{qsoptex}

\section{Python bindings}

As a programming language, \texttt{python} is widely used in the scientific community.
While its abstractions make it very easy to read and write, this comes at the expense of performance, which is usually abysmal compared to other languages, such as \texttt{c++}.
There are several \texttt{python} implementations.
The most popular one is \texttt{CPython}, which is also the one this project targets, but some alternatives include \texttt{Jython}, \texttt{IronPython} and \texttt{PyPy} \cite{man:python-implementations}.

The \texttt{CPython} implementation also allows for extensions.
The Python API defines a set of functions, macros and variables that provide access to the aspects of the Python run-time system.
To access them, it is sufficient to include the header file \texttt{Python.h}.
In other words, Python bindings (or extensions) are a way to write \texttt{c++} code that integrates seamlessly and can called from a \texttt{python} program.
Instead of using the standard \texttt{python} API, though, the much more convenient \texttt{pybind11} library is used \cite{man:pybind11}.
The main appeal of \texttt{pybind11} is that it exposes \texttt{c++} classes and functions to \texttt{python} with a very minimalistic syntax.
It also provides a rich built-in type conversion system by inferring type information using compile-time introspection.
The user can extend it even further to support custom types.

In the context of \dlinear, the main functionalities of the solver, as well as some of the most valuable data manipulation classes, have been exposed in a \texttt{python} package called, unsurprisingly, \pydlinear.
The bindings are defined in the \texttt{pydlinear.cpp} file (\autoref{code:example_python_bindings}).
The \texttt{setup.py}, with the configuration options provided by the \texttt{pyproject.toml} file, defines the steps needed for building the package to make it available to the \texttt{python} interpreter.
While the ideal scenario would be to build it on the target machine, it is possible to include a pre-build version called a \textit{wheel}.
This particular archive contains all the shared libraries that make up the package, and provided that the architecture of both the distributor's machine and the user's machine is the same, it can be used without any need for compilation by the final user.

\lstinputlisting[language=c++,frame=single,showstringspaces=false,caption={Simplified example of \pydlinear's bindings},captionpos=b,label={code:example_python_bindings}]{code/example\_python\_bindings.cpp}

The package has been uploaded to the \texttt{PyPI} repository, making it available to anyone with a \texttt{Linux x86\_64} machine and \texttt{python 3.10}.
It can be installed with the command

\begin{lstlisting}[language=bash,frame=single,showstringspaces=false]
$ pip install -i https://test.pypi.org/simple/ pydlinear==0.0.1
\end{lstlisting}
Once installed, the solver's APIs are accessible by importing the \pydlinear package in the script (\autoref{code:pydlinear.py}).

\lstinputlisting[language=python,frame=single,showstringspaces=false,caption={Minimal python program using \pydlinear},captionpos=b,label={code:pydlinear.py}]{code/pydlinear.py}

\section{Targets}

\bazel uses the concept of \textit{targets} to define the building blocks of a workspace.
A target can be a library, an executable, a test, a script or a data file.
What follows is a synopsis of the most essential targets used in \dlinear and their purpose.

\begin{itemize}
        \item \textbf{//dlinear}: supports both \textit{build} and \textit{run} actions.
              It is the project's main target, used by the \texttt{benchmark} and \texttt{test} targets.
              It produces the fully static binary \texttt{dlinea5}, located in the \texttt{bazel-bin/dlinear} folder.
        \item \textbf{//benchmark}: supports both \textit{build} and \textit{run} actions.
              It creates a binary that wraps the \texttt{dlinear} and measures the time it takes to run the program on a set of problems with a given solver and precision.
              Both parameters can be configured using a \texttt{benchmark.conf} file.
              By default, it checks all the files in the \textit{benchmark/smt2} folder.
        \item \textbf{//test/...}: supports the \textit{test} action.
              It runs all the tests in the \dlinear.
              The logs will be saved in the \texttt{bazel-testlogs} folder.
        \item \textbf{//pydlinear}: supports the \textit{build} action.
              It creates the \texttt{pydlinear} extension.
              It can be a dependency within \bazel for other \texttt{python} targets.
\end{itemize}
