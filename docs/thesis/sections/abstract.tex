\chapter*{Abstract}
This thesis presents a study and improvements over the \dlinearfour SMT delta solver.
We will start by introducing the SMT problems, their applications and some state-of-the-art algorithms used to solve them efficiently.
Then we delve deep into the work done by the PhD student Martin, the \dlinearfour solver, in turn, inspired by the dReal4 solver.

We aim to provide a detailed description of the software, its architecture and the choices made to ensure correctness and efficiency.
The objective is to build upon what is already present while improving performance and general code quality where possible.
It must be noted that the codebase was already pretty well written.
Still, some of it can be improved by introducing modern C++17 features.
Code repetition can be minimized in multiple places by making the class extend from a common superclass.
Even the use of global non-reentrant configurations can be made more manageable by using an appropriate design patter.

Another area of interest is the build system itself, for it is a crucial part of the software development process.
The original codebase required the manual installation of most of its dependencies, without leveraging all the features \bazel provides.
This has been changed, allowing for a better and more reproducible build process, which also include testing and benchmarking.

Most other software of this kind greatly benefits from having a Python interface, which allows for a more user-friendly experience without the hassle of having to write C++ code.
Hence I added the package \pydlinear, which provides the required Python bindings to the \dlinear C++ library with a negligible cost in performance.

Finally, in the appendices, we will present the results of our experiments with the use of benchmarks to allow the comparison with other solvers.
